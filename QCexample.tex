\section{Quantum computer's calculation example}
The following lines of codes need to be added in your Python script in order to connect to the Triangulum Quantum Computer:

\begin{lstlisting}[language=bash]
username="Ulysses"
password="Odissey"
config.configure_ip("spinq.fisgeo.unipg.it")
config.configure_port(55444)
config.configure_account(username, password)
...
\end{lstlisting}

\noindent  The web address  \textit{spinq.fisgeo.unipg.it} points to the workstation's IP address connected to the Triangulum through the port 55444. \\

\noindent You need to ask your system administrator to create an \textit{username} and \textit{password} in the quantum computer.  
Due to some translation error in the interface, it is advisable to use the same entries for \textit{Account} and \textit{Name} while creating your local account. The variable  \textit{username} corresponds to the entry \textit{Account}.

\subsection{Example: GHZ state}
\begin{lstlisting}[language=bash]
from spinqit import get_nmr, get_compiler, Circuit, NMRConfig
from spinqit import H, CX

username="Ulysses"
password="Odissey"

engine = get_nmr()
comp = get_compiler("native")

circ = Circuit()

q = circ.allocateQubits(3)
circ << (H, q[0])
circ << (CX, (q[0], q[1]))
circ << (CX, (q[1], q[2]))

exe = comp.compile(circ, 0)
config = NMRConfig()
config.configure_shots(1)
config.configure_ip("spinq.fisgeo.unipg.it")
config.configure_port(55444)
config.configure_account(username, password)
config.configure_task("task1", "GHZ")


result = engine.execute(exe, config)
print(result.probabilities)
\end{lstlisting}
